\documentclass{article} % For LaTeX2e
\usepackage{graphicx} %\layout: page width = 397pts = 5.51inches = 14cm
\usepackage{psfrag} 

\begin{document}

\section*{Version 1: Export using matlab menus}
\begin{center}
\includegraphics[width=397pt]{demo1.eps}
\end{center}
\pagebreak

\section*{Version 2: Export using print}
\begin{center}
\includegraphics[width=397pt]{demo2.eps}
\end{center}
\pagebreak

\section*{Version 3: Setting figure page size}
\begin{center}
\includegraphics[width=397pt]{demo3.eps}
\end{center}
\pagebreak

\section*{Version 4: Setting axis sizes}
\begin{center}
\includegraphics[width=397pt]{demo4.eps}
\end{center}
\pagebreak

\section*{Version 5: Setting line and font style}
\begin{center}
\includegraphics[width=397pt]{demo5.eps}
\end{center}
\pagebreak

\section*{Version 6: Using PSfrag on right xlabel}
\begin{center}
\psfrag{omega = 2 pi c/lambda}[l][c]{$\omega = \frac{2 \pi c}{\lambda}$}
\includegraphics[width=397pt]{demo6.eps}
\end{center}
\pagebreak

\section*{Version 7: My code and PSfrag on all labels}
\begin{center}
\psfrag{omega = 2 pi c/lambda}[l][c]{$\omega = \frac{2 \pi c}{\lambda}$}
\psfrag{y}[l][c]{$y$}
\includegraphics[width=397pt]{demo7.eps}
\end{center}

\end{document}
